\chapter{Printing (with {\tt CUPS})}
\section{Configuring the printer}

1.) Open your favorite internet browser and go to \url{http://localhost:631/}

2.) Select administration from the top black line. If required, type root as username, and give your root password.

3.) Here you can do almost everything in connection with printing. In our example, we will add a new local printer.

4.) Click \textit{Add Printer}, type a name and optionally fill the \textit{Location} and \textit{Description} lines.

5.) Then select \textit{Device}, in most cases that will be {\tt Parallel Port #1}.

6.) On the next page, select your vendor and your printer type. 

7.) That's it!

\section{Obtaining and installing the suitable driver (optional)}

If your vendor or printer type isn't listed in the wizard (this was the case with my \textit{Okidata Okipage 6e}), you have to download the driver from \url{http://linuxprinting.org/}.

Here, I will show only an example, too. On the right side, select \textit{Printer Listings}. Then we select our \textit{Okidata} vendor and \textit{Okipage 6e} type. At the result page select \textit{download PPD}. After downloading we got a {\tt Okidata-Okipage\_6e\-hpijs.ppd}.
 
Save the PPD file in the directory {\tt /usr/share/cups/model/}. The PPD file does not need to be executable, but it should be world-readable and should have the file name extension ".ppd".

Then restart the cups service: {\tt su -c 'service cups restart'}.
 
The driver installation is now completed, you can add your printer now via the web interface.
 
(A good howto can be found at \url{http://linuxprinting.org/cups-doc.html}, too.)

\chapter{Applications and their configuration}

\section{Generally}
Where it is available, the documentation of a package can be found in {\tt /usr/share/doc/\$packagename-\$packagever/}. 

\section{lmsensors}

lmsensors is a hardware monitoring tool which is able to read thermal and voltage values from the sensor chip of your motherboard.
\textbf{Before running {\tt sensors} you have to run {\tt sensors-detect} as root to initialize(the i2c-dev kernel modul also needed for this).} This will autodetect your hardware and give the information, what kernel modules you will need to get {\tt sensors} work properly, and how to autoload them at init. Also look at chapter \ref{chap:init}

\section{tetex and tetex-doc}

One can find the tetex distribution in the extra/xapps category. Note that it is shipped without the documentation, only a very good tutorial can be found in /usr/share/texmf/doc.
If you need the whole doc directory, use xbit's repository:
\url{http://wiki.frugalware.org/index.php/DRs}

\section{apache/httpd}

\textit{How to configure Apache (with SSL support)}

0) These steps will require root privileges, so use {\tt su -} to get a root shell.

1) The Apache server isn't started by default. We can change this with the

{\tt service httpd add}

command.

2) We don't want to reboot, so we start it manually:

{\tt root@vmhome:~# service httpd start}

{\tt Starting Apache web server (no SSL)        [  OK  ]}

3) If you want SSL support:

{\tt root@vmhome:~# cd /etc/httpd/conf/}

{\tt root@vmhome:/etc/httpd/conf# sh mkcert.sh}

{\tt Signature Algorithm ((R)SA or (D)SA) [R]:}

Here firstly we can accept the default RSA signature algorithm.

Then have to fill out some fields. There are quite a few fields but you can leave some blank.
If you enter '.', the field will be left blank.

{\tt 1. Country Name             (2 letter code) [XY]:}

Then we give the 2 letter code of our contry (for example US)

{\tt 2. State or Province Name   (full name)     [Snake Desert]:}

We type our state.

{\tt 3. Locality Name            (eg, city)      [Snake Town]:}

The name of our city.

{\tt 4. Organization Name        (eg, company)   [Snake Oil, Ltd]:}

Our organization's name.

{\tt 5. Organizational Unit Name (eg, section)   [Webserver Team]:}

Our section's name.

{\tt 6. Common Name              (eg, FQDN)      [www.snakeoil.com]:}

\textbf{Important}: Give here the real address, othervise you'll get warnings in your browser!

{\tt 7. Email Address            (eg, name@FQDN) [www@snakeoil.dom]:}

I usually give here the email address of the webmaster. (\textit{webmaster@domain.com})

{\tt 8. Certificate Validity     (days)          [365]:}

Usually one year will be good.

Then, we should choose the version of our certificate:

{\tt Certificate Version (1 or 3) [3]:}

The default 3 will be good, so just hit enter.

In the next step we can encrypt our private key:

{\tt Encrypt the private key now? [Y/n]:}

The keys will \textit{not} be readable by users, so we can leave this step out.

So the following files are created: {\tt /etc/httpd/conf/ssl.key/server.key} (keep this file private!), {\tt /etc/httpd/conf/ssl.crt/server.crt} and {\tt /etc/httpd/conf/ssl.csr/server.csr}.

4) Now we should restart apache:

{\tt root@vmhome:~# service httpd restart}

{\tt Stopping Apache web server [  OK  ]}

{\tt Starting Apache web server [  OK  ]}

5) Then we can check if the task was successfull:

{\tt vmiklos@vmhome:~\$ elinks https://localhost/}

This should show the default homapage, received via SSL :)

\chapter{Sound Configuration}
\section{Configuring the sound card}

Frugalware uses the Advanced Linux Sound Architecture (ALSA) subsystem for sound cards. For older applications, the Open Sound System (OSS) compatibility modules are loaded, but Frugalware do \textit{not} contain native OSS support.

Finding and loading the necessary module for your sound card is fairly simple. Mostly the same as setting up your network card. During every boot, the hotplug scripts will setup your sound card, but, of course, you can take the automatically loaded module to blacklist, and load manually your module by editing {\tt /etc/rc.d/rc.modules}.

\section{Volume configuration with {\tt alsamixer}}

By default, your sound card is muted. You can use {\tt alsamixer} to unmute your card. Use the {\tt <} and {\tt >} keys to unmute a channel, {\tt up} and {\tt down} keys to set the volume and {\tt left} or {\tt right} keys to switch to another channel.

You can exit from {\tt alsamixer} by hitting the {\tt Esc} key. On shutdown, Frugalware saves your settings, but you can store them right now by using the command {\tt su -c 'service alsa restart'}.
